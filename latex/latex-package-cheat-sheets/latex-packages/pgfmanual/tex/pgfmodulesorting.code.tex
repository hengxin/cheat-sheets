% Copyright 2008 by Till Tantau
%
% This file may be distributed and/or modified
%
% 1. under the LaTeX Project Public License and/or
% 2. under the GNU Public License.
%
% See the file doc/generic/pgf/licenses/LICENSE for more details.

\ProvidesFileRCS[v\pgfversion] $Header: /cvsroot/pgf/pgf/generic/pgf/modules/pgfmodulesorting.code.tex,v 1.1 2008/11/12 23:39:37 tantau Exp $

%
% This file defines commands for sorting data. This is useful for 3d plots,
% but also for drawing things in the correct order.
%


\newcount\pgfsortingbuckets


% Init the sorting
%
% #1 = number of buckets
%
% Description:
%
% This macro inits a sorting. At any time there can be only one active
% sorting. When you init the sorting, #1 should be a number of
% buckets. You can then call \pgfsortinginsert to insert code into the
% buckets. Once you are done, calling \pgfsortingexecute will execute
% first the code in buckets 0, then the code in bucket 1, and so on to
% bucket number #1-1.
%
\def\pgfsortinginit#1{
  \pgfsortingbuckets#1\relax%
  \c@pgf@counta=0\relax
  \pgfutil@loop
  \ifnum\c@pgf@counta<\c@pgf@countb\relax%
    \expandafter\global\expandafter\let\csname pgf@sort@\the\c@pgf@counta\endcsname\pgfutil@empty%
    \advance\c@pgf@counta by1\relax%
  \pgfutil@repeat
}

% Insert code into some bucket
% 
% #1 = the bucket
% #2 = the code
% 
\def\pgfsortinginsert#1#1{
  \expandafter\let\expandafter\pgf@temp\csname pgf@lib@sort@#1\endcsname%
  \expandafter\def\expandafter\pgf@temp\expandafter{\pgf@temp#2}%
  \expandafter\global\expandafter\let\csname pgf@lib@sort@\the\c@pgf@counta\endcsname\pgf@temp%
}


% After all code has been inserted into the buckets, you can call
% this method to execute the code in the order of the buckets.
\def\pgfsortingexecute{
  \c@pgf@countb=\pgfoovalueof{buckets}\relax
  \c@pgf@counta=0\relax
  \pgfutil@loop
  \ifnum\c@pgf@counta<\c@pgf@countb\relax%
    \csname pgf@lib@sort@\the\c@pgf@counta\endcsname%
    \advance\c@pgf@counta by1\relax%
  \pgfutil@repeat
}


\endinput
